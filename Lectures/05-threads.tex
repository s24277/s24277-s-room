\documentclass[aspectratio=169]{beamer}

\usepackage[utf8]{inputenc}
\usepackage{textcomp}
\usepackage[polish]{babel}
\usepackage{amsthm}
\usepackage{graphicx}
\usepackage[T1]{fontenc}
\usepackage{scrextend}
\usepackage{hyperref}
\usepackage{xcolor}
\usepackage{geometry}
\usepackage{listings}
\usepackage{epigraph}

\renewcommand{\epigraphsize}{\scriptsize}

\usetheme{-bjeldbak/beamerthemebjeldbak}

\definecolor{xbarcolor}{HTML}{000000}
\setbeamercolor{lower separation line head}{bg=xbarcolor}
\setbeamercolor{lower separation line foot}{bg=xbarcolor}

\title{Podstawy programownia (w języku C++)}
\subtitle{Wątki}
\author{Marek Marecki}
\institute{Polsko-Japońska Akademia Technik Komputerowych}

\lstset{basicstyle=\ttfamily\color{black},
columns=fixed,
escapeinside={[*}{*]},
inputencoding=utf8,
extendedchars=true,
moredelim=**[is][\color{red}]{@}{@},
moredelim=**[is][\color{gray}]{`}{`},
moredelim=**[is][\color{olive}]{$}{$}}

\begin{document}

{%
    \setbeamertemplate{headline}{}
    \frame{\titlepage}
}

\section{Wstęp}

\begin{frame}
    \frametitle{Po co wątki w C++?}

    Wątki w C++ są narzędziem, które można wykorzystać do kilku celów. Nie
    znaczy to, że są najlepszym możliwym rozwiązaniem! Są jedynie podstawą,
    która umożliwia stworzenie i wykorzystanie bardziej eleganckich mechanizmów.

    \vspace{1em}

    Wątki są również sposobem na wprowadzenie do programów
    \textbf{współbieżności}~(ang. \emph{concurrency}) i
    \textbf{równoległości}~(ang. \emph{parallelism}).
\end{frame}

\begin{frame}
    \frametitle{Współbieżność (ang. concurrency)}

    Wykonywanie przez program kilku zadań naraz (ale niekoniecznie
    \emph{jednocześnie}).

    \vspace{1em}

    Iluzja wykonywania przez program kilku zadań naraz może być zapewniona,
    dzięki szybkiemu przełączaniu się między częściowo wykonanymi zadaniami.

    \vspace{1em}

    Przykład: program kopiujący $k$ plików. Program może skopiować plik $n$ w
    całości, po czym przejść do kopiowania pliku $n+1$ i działać sekwencyjnie.
    Może też kopiować fragmenty plików od $n$ do $n+k$ i działać współbieżnie
    (dzięki temu małe pliki mogą być skopiowane w całości szybciej).
\end{frame}

\begin{frame}
    \frametitle{Równoległość (ang. parallelism)}

    Wykonywanie przez program kilku zadań \emph{jednocześnie} (w tym samym
    kwancie czasu).

    \vspace{1em}

    Równoległość możliwa jest do osiągnięcia dzięki wykorzystaniu w programie
    wielu procesorów (na tym samym fizycznym komputerze, lub w systemie
    rozproszonym).

    \vspace{1em}

    Przykład: program kopiujący $k$ plików. Program może podzielić listę plików
    do skopiowania na $n$ procesorów, i każdemu przekazać do obsłużenia $k/n$
    plików.
\end{frame}

\begin{frame}
    \frametitle{Współbieżność \emph{vs} równoległość}

    Współbieżność \emph{nie zawsze} oznacza równoległość.\\
    Równoległość \emph{zawsze} oznacza współbieżność.

    \vspace{1em}

    Współbieżność jest sposobem na modelowanie zachowania i podział zadań
    wewnątrz programu. Równoległość jest sposobem na pełniejsze wykorzystanie
    zasobów systemu.
\end{frame}

\section{Wątki w C++}

\begin{frame}[fragile]
    \frametitle{\texttt{std::thread}}
    \framesubtitle{Wątek czyli \texttt{std::thread}}

    Wątki tworzy się za pomocą typu \texttt{std::thread} z nagłówka
    \texttt{<thread>}.

    \vspace{1em}

    Utworzenie wątku oznacza rozpoczęcie współbieżnego wykonywania funkcji,
    która została przekazana do konstruktora typu \texttt{std::thread}
    {\small
    \begin{lstlisting}
    auto a_thread = std::thread{
          some_function
        , 42
        , "Hello, World!"
    };
    \end{lstlisting}}
\end{frame}

\begin{frame}[fragile]
    \frametitle{,,Dołączanie'' wątków}
    \framesubtitle{\texttt{std::thread}}

    Za pomocą funkcji składowej \texttt{std::thread::join} można ,,dołączyć''
    wątek reprezentowany jakąś zmienną do wątku, z którego ta funkcja jest
    wywołana.

    \vspace{1em}

    {\scriptsize
    \begin{lstlisting}
    auto errand_boy = std::thread{run_some_errands, std::move(errands)};
    errand_boy.join();  $// wait until the errand boy returns...$
    \end{lstlisting}}

    \vspace{1em}

    Wątek wywołujący funkcję \texttt{std::thread::join} jest zablokowany do
    momentu aż wątek ,,dołączany'' nie zakończy działania.
\end{frame}

\begin{frame}[fragile]
    \frametitle{,,Odcinanie'' wątków}
    \framesubtitle{\texttt{std::thread}}

    Za pomocą funkcji składowej \texttt{std::thread::detach} można ,,odciąć''
    wątek i uniemożliwić jego dołączenie do innego wątku. Pozwala to odciętemu
    wątkowi działać nawet jeśli żaden inny wątek nie posiada jego
    ,,uchwytu''\footnote{\emph{thread handle}}:

    \vspace{1em}

    {\scriptsize
    \begin{lstlisting}
    auto fire_and_forget = std::thread{launch_missile, std::move(target)};
    fire_and_forget.detach();  $// too late to change our mind now...$
    \end{lstlisting}}
\end{frame}

\begin{frame}
    \frametitle{Zadanie: drukowanie napisów}
    \framesubtitle{\texttt{std::thread}}
    \label{lecture_exercise_0}

    Napisz program, który uruchomi 42 wątki drukujące napis ,,\texttt{Hello,
    X!}''.

    \texttt{X} musi być różne dla każdego wątku (mogą to być różne imiona, różne
    liczby, itp.).

    Wykorzystaj funkcję \texttt{std::thread::detach}.

    \vspace{1em}

    Kod w pliku \texttt{src/s05-print-thread.cpp}
\end{frame}

\begin{frame}
    \frametitle{Zadanie: drukowanie grup napisów}
    \framesubtitle{\texttt{std::thread}}
    \label{lecture_exercise_1}

    Napisz program, który uruchomi 42 wątki drukujące napis ,,\texttt{Hello,
    X!}''.

    \texttt{X} musi być różne dla każdego wątku (mogą to być różne imiona, różne
    liczby, itp.).

    Wątki mają być uruchamiane grupami po 6 naraz. Wykorzystaj funkcję
    \texttt{std::thread::join} do czekania na zakończenie wątku.

    \vspace{1em}

    Kod w pliku \texttt{src/s05-print-thread-group.cpp}
\end{frame}

\begin{frame}
    \frametitle{Sekcje krytyczne}

    Sekcja krytyczna to fragment kodu, który może być wykonywany przez jeden
    wątek naraz. Takie fragmenty kodu powinny być zabezpieczone przed
    współbieżnym wykonaniem.

    \vspace{1em}

    \textbf{Jak rozpoznać, że coś jest sekcją krytyczną?}

    Sekcja krytyczna dotyka (manipuluje, odczytuje, itd.) danych, które widoczne
    są z potencjalnie wielu wątków.
\end{frame}

\begin{frame}[fragile]
    \frametitle{Zabezpieczenie sekcji krytycznych}
    \framesubtitle{Sekcje krytyczne}

    Najprościej jest wykorzystać typy \texttt{std::mutex}\footnote{\emph{mutex}
    od \emph{mutual exclusion}} do zabezpieczenia danych, oraz
    \texttt{std::lock\_guard} lub \texttt{std::unique\_lock} do zablokowania
    wątku w oczekiwaniu na dostęp do sekcji krytycznej.

    \vspace{1em}

    Zabezpieczenie danych:
    {\scriptsize
    \begin{lstlisting}
    auto foo = some_type{};
    std::mutex foo_mtx;  $// mutex for foo$
    \end{lstlisting}}

    Oczekiwanie na dostęp do sekcji krytycznej:
    {\scriptsize
    \begin{lstlisting}
    std::lock_guard<std::mutex> lck { foo_mtx };
    $// or...$
    std::unique_lock<std::mutex> lck { foo_mtx };
    \end{lstlisting}}
\end{frame}

\begin{frame}[fragile]
    \frametitle{Zadanie: lista zadań}
    \framesubtitle{\texttt{std::thread}}
    \label{lecture_exercise_2}

    Napisz program, w którym uruchomisz cztery wątki pobierające ze
    wspólnej\footnote{potrzebna będzie tu referencja lub wskaźnik} kolejki
    \texttt{std::queue} liczby do wydrukowania. Pobieranie liczb powinno być
    sekcją krytyczną.

    Każdemu z wątków przydziel ID (przekazywane jako argument podczas tworzenia
    wątku). Drukowane linie powinny mieć następujący format:

    {\scriptsize
    \begin{lstlisting}
    from thread @THREAD-ID@: @NUMBER-TO-PRINT@
    \end{lstlisting}}

    Niech wątki zakończą działanie kiedy wykryją, że wektor z liczbami do
    wydrukowania jest pusty.

    \vspace{1em}

    Kod w pliku \texttt{src/s05-job-queue.cpp}
\end{frame}

\section{Komunikacja między wątkami}

\begin{frame}[fragile]
    \frametitle{Zadanie: śpiące wątki}
    \framesubtitle{\texttt{std::thread}}
    \label{lecture_exercise_3}

    {\small
    Uruchom 4 wątki pobierające rzeczy do wydrukowania ze wspólnej kolejki (jak
    w poprzednim zadaniu). Niech wątki kończą pracę kiedy dostaną pusty
    \texttt{std::string} i drukują \texttt{thread X exiting} kiedy kończą pracę.
    Kiedy wątek wykryje, że kolejka jest pusta niech zwolni mutex i pójdzie spać
    na losową wartość milisekund między 10, a 100.

    \vspace{1em}

    W funkcji \texttt{main()} odczytuj napisy ze standardowego strumienia
    wejścia i odczytane napisy odkładaj na kolejkę, z której czytają wątki
    drukujące.
    Zakończ pętlę po otrzymaniu 4 pustych napisów.

    \vspace{1em}

    Kod w pliku \texttt{src/s05-sleeping-threads.cpp}}
\end{frame}

\begin{frame}
    \frametitle{Oczekiwanie na, i sygnalizacja warunku}
    \framesubtitle{\texttt{std::condition\_variable}}

    Kod taki jak w zadaniu ze śpiącymi wątkami jest nieefektywny, mało
    czytelny, i - powiedzmy to jasno - kiepski. Jeśli \emph{oczekujemy} na
    wystąpienie jakiegoś warunku (np. nadejście zadania) to dużo lepszy jest
    kod, który realizuje to w sposób jawny.

    \vspace{1em}

    Typ \texttt{std::condition\_variable} z nagłówka
    \texttt{<condition\_variable>} pozwala to zrealizować w prosty sposób.
\end{frame}

\begin{frame}
    \frametitle{Przygotowanie}
    \framesubtitle{\texttt{std::condition\_variable}}

    Żeby móc oczekiwać na warunek potrzebujemy następujących rzeczy:
    \begin{enumerate}
        \item zmiennej typu \texttt{std::condition\_variable}, której będziemy
            używać do oczekiwania i sygnalizacji warunku
        \item zmiennej typu \texttt{std::mutex}, na której będziemy blokować
            nasz condition variable
        \item dodatkowej zmiennej przechowującej dane, które
            powinniśmy sprawdzić po zajściu warunku (np. kolejki rzeczy do
            wydrukowania)
    \end{enumerate}
\end{frame}

\begin{frame}[fragile]
    \frametitle{Przygotowanie (c.d.)}
    \framesubtitle{\texttt{std::condition\_variable}}

    {\small
    \begin{lstlisting}
    $// this is our communication channel$
    std::queue<std::string> items_to_print;

    $// the mutex protects it from concurrent access$
    std::mutex mtx;

    $// the condition_variable is used to signal
    // arrival of new data$
    std::condition_variable cv;
    \end{lstlisting}}
\end{frame}

\begin{frame}[fragile]
    \frametitle{Oczekiwanie na zajście warunku}
    \framesubtitle{\texttt{std::condition\_variable}}

    Oczekiwanie na zajście warunku jest realizowane funkcjami składowymi
    \texttt{std::condition\_variable::wait} (czekanie w nieskończoność) i
    \texttt{std::condition\_variable::wait\_for} (czekanie przez określony
    czas).

    \vspace{1em}

    Obie funkcje żądają przekazania im jako argumentu wartości typu
    \texttt{std::unique\_lock} zablokowanej na muteksie, który chroni dane
    będące kanałem komunikacji:

    {\scriptsize
    \begin{lstlisting}
    $// lock the mutex$
    std::unique_lock<std::mutex> lck { mtx };
    \end{lstlisting}}
\end{frame}

\begin{frame}[fragile]
    \frametitle{Oczekiwanie na zajście warunku (c.d.)}
    \framesubtitle{\texttt{std::condition\_variable}}

    Jeśli jesteśmy w stanie pozwolić sobie na czekanie w nieskończoność możemy
    użyć prostego \texttt{wait}:
    {\scriptsize
    \begin{lstlisting}
    $// unlock the mutex, block calling thread, and
    // wait for a notification (ad inifinitum)$
    cv.wait(lck);
    \end{lstlisting}}

    Jeśli jednak w określonym czasie musimy podjąć jakieś działanie niezależnie
    od tego czy otrzymaliśmy komunikat czy nie, należy użyć \texttt{wait\_for}:
    {\scriptsize
    \begin{lstlisting}
    $// unlock the mutex, block calling thread, and
    // wait for a notification for about 1 second$
    cv.wait(lck, std::chrono::seconds{1});
    \end{lstlisting}}
\end{frame}

\begin{frame}[fragile]
    \frametitle{Sygnalizacja warunku}
    \framesubtitle{\texttt{std::condition\_variable}}

    {\small
    \begin{lstlisting}
    $// lock the mutex and modify communication channel...$
    {
        std::lock_guard<std::mutex> lck { mtx };
        items_to_print.push("Hello, World!");
    }

    $// notify just one of the waiting threads...$
    cv.notify_one();

    $// ...or notify all of them$
    cv.notify_all();
    \end{lstlisting}}
\end{frame}

\begin{frame}[fragile]
    \frametitle{Zadanie: śpiące wątki 2}
    \framesubtitle{\texttt{std::thread}}
    \label{lecture_exercise_4}

    {\small
    Uruchom 4 wątki pobierające rzeczy do wydrukowania ze wspólnej kolejki (jak
    w poprzednim zadaniu). Niech wątki kończą pracę kiedy dostaną pusty
    \texttt{std::string} i drukują \texttt{thread X exiting} kiedy kończą pracę.

    Niech wątki oczekują na rzeczy do wydrukowania z wykorzystaniem funkcji
    \texttt{std::condition\_variable::wait}.

    \vspace{1em}

    W funkcji \texttt{main()} odczytuj napisy ze standardowego strumienia
    wejścia i odczytane napisy odkładaj na kolejkę, z której czytają wątki
    drukujące.
    Zakończ pętlę po otrzymaniu 4 pustych napisów.

    \vspace{1em}

    Kod w pliku \texttt{src/s05-sleeping-threads-2.cpp}}
\end{frame}

\begin{frame}[fragile]
    \frametitle{Zadanie: śpiące wątki 3}
    \framesubtitle{\texttt{std::thread}}
    \label{lecture_exercise_5}

    Tak jak w ,,śpiące wątki 2'', ale z użyciem typu \texttt{itp::channel} z
    repozytorium szablonowego.

    \vspace{1em}

    Kod w pliku \texttt{src/s05-sleeping-threads-3.cpp}
\end{frame}

\begin{frame}[fragile]
    \frametitle{Zadanie: ping-pong}
    \framesubtitle{\texttt{std::thread}}
    \label{lecture_exercise_6}

    {\small
    Uruchom dwa wątki - jeden z identyfikatorem ,,ping'', drugi ,,pong''.
    Niech oba wątki wymieniają się liczbą całkowitą ,,odbijając'' ją między
    sobą.

    Wątek powinien relizować w pętli następującą logikę:
    \begin{enumerate}
        \item czekać na komunikat o otrzymaniu liczby
        \item wydrukować liczbę poprzedzoną swoim identyfikatorem (np.
            \texttt{ping 7})
        \item zwiększyć liczbę o losową wartość między 1, a 42
        \item poinformować drugi wątek o tym, że teraz jego kolej
        \item jeśli liczba była większa niż 1024 zakończyć działanie
    \end{enumerate}

    Pierwszą liczbę wyślij do wątku ,,ping'' z funkcji \texttt{main()}.

    \vspace{1em}

    Kod w pliku \texttt{src/s05-ping-pong.cpp}}
\end{frame}

\section{Podsumowanie}

\begin{frame}
    \frametitle{Co nowego?}
    \frametitle{Podsumowanie}

    Student powinien umieć:

    \begin{enumerate}
        \item wykorzystać wątki do wprowadzenie współbiezności i równoległości w
            programie
        \item wytłumaczyć do czego służą typy \texttt{std::mutex},
            \texttt{std::unique\_lock}, i \texttt{std::lock\_guard}
        \item wytłumaczyć czym jest sekcja krytyczna
    \end{enumerate}
\end{frame}

\begin{frame}
    \frametitle{Zadania}
    \framesubtitle{Podsumowanie}

    Zadania znajdują się na slajdach
    \ref{lecture_exercise_0},
    \ref{lecture_exercise_1},
    \ref{lecture_exercise_2},
    \ref{lecture_exercise_3},
    \ref{lecture_exercise_4},
    \ref{lecture_exercise_5},
    \ref{lecture_exercise_6}.
\end{frame}

\end{document}
